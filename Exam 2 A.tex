\documentclass[11pt]{article}

\newcommand{\NUMBER}{1110}
\newcommand{\TERM}{Summer 2022}
\newcommand{\DOCUMENT}{Exam 2}
\newcommand{\VERSION}{A}

\usepackage{ifpdf}
\usepackage{calc}
\usepackage{palatino}
\usepackage{xspace}
\usepackage{fancyhdr}
\usepackage{graphicx}
\usepackage{color}
\usepackage{pifont}
\usepackage{enumitem}
\usepackage{amsmath}
\usepackage{esint}
\usepackage[e]{esvect}
\usepackage{multicol}
\usepackage{ulem}
\usepackage{changepage}
\usepackage{lastpage}
\usepackage{wrapfig}
\usepackage{hyperref}

\oddsidemargin 0.0in
\evensidemargin 0.0in
\headheight 0.2in
\textwidth 6.6in
\textheight 8.7in
\topmargin -0.25in

\pagestyle{fancyplain}
\renewcommand{\plainheadrulewidth}{0.4pt}
\renewcommand{\headrulewidth}{0.4pt}
\chead{\fancyplain{}{\DOCUMENT\ (cont.)}}
\rhead{\fancyplain{}{\thepage\ of \pageref{LastPage}}}
\cfoot{\fancyplain{}{Version \VERSION}}

\fancypagestyle{firststyle}{
\fancyhf{}
\fancyfoot[R]{\footnotesize Page \thepage\ of \pageref{LastPage}}
\renewcommand{\headrulewidth}{0.0pt}
}

\parindent 0.0in
\parskip 1ex

\setlength\jot{12pt}
\setlength{\fboxsep}{10pt}

\newcommand{\tw}{\textwidth}

\newcommand{\AM}  {\mbox{\bf A}\xspace}
\newcommand{\BM}  {\mbox{\bf B}\xspace}
\newcommand{\CM}  {\mbox{\bf C}\xspace}
\newcommand{\DM}  {\mbox{\bf D}\xspace}
\newcommand{\OM}  {\mbox{\bf O}\xspace}
\newcommand{\UM}  {\mbox{\bf 1}\xspace}
\newcommand{\ABM} {\mbox{\bf AB}\xspace}
\newcommand{\BAM} {\mbox{\bf BA}\xspace}
\newcommand{\AMI} {\mbox{{\bf A}$^{-1}$}\xspace}
\newcommand{\LM}  {\mbox{\boldmath$\lambda$}\xspace}

\newcommand{\A}	{\mbox{$\vv{\mathbf A}$}\xspace}
\newcommand{\B}	{\mbox{$\vv{\mathbf B}$}\xspace}
\newcommand{\C}	{\mbox{$\vv{\mathbf C}$}\xspace}
\newcommand{\D} {\mbox{$\vv{\mathbf D}$}\xspace}
\newcommand{\E} {\mbox{$\vv{\mathbf E}$}\xspace}
\newcommand{\R}	{\mbox{$\vv{\mathbf R}$}\xspace}
\renewcommand{\O}	{\mbox{$\mathbf{O}$}\xspace}
\renewcommand{\v}	{\mbox{$\vv{\mathbf{v}}$}\xspace}
\newcommand{\s}    {\mbox{$\mathbf{s}$}\xspace}
\newcommand{\n}    {\mbox{$\mathbf{n}$}\xspace}
\newcommand{\x}   {\mbox{$\vv{\mathbf x}$}\xspace}
\newcommand{\g}   {\mbox{$\vv{\mathbf g}$}\xspace}
\newcommand{\F}   {\mbox{$\vv{\mathbf F}$}\xspace}
\renewcommand{\L}    {\mbox{$\vv{\mathbf L}$}\xspace}
\newcommand{\tauv}  {\mbox{\boldmath$\vv{\tau}$}\xspace}
\newcommand{\p}	{\mbox{$\vv{\mathbf p}$}\xspace}
\renewcommand{\r}	{\mbox{$\vv{\mathbf r}$}\xspace}
\renewcommand{\a}  {\mbox{$\vv{\mathbf a}$}\xspace}
\newcommand{\w}  {\mbox{$\vv{\mathbf w}$}\xspace}
\newcommand{\Svec}  {\mbox{$\vv{\mathbf S}$}\xspace}
\newcommand{\W}  {\mbox{$\vv{\mathbf W}$}\xspace}
\newcommand{\del}  {\mbox{\boldmath$\nabla$}\xspace}
\newcommand{\ov}   {\mbox{\boldmath $\vv{\omega}$}\xspace}
\newcommand{\vdot} {\mbox{$\mathbf{\dot{v}}$}\xspace}
\newcommand{\rdot} {\mbox{$\mathbf{\dot{r}}$}\xspace}
\newcommand{\rddot} {\mbox{$\mathbf{\ddot{r}}$}\xspace}

\newcommand{\X} {\mbox{$X$}\xspace}
\newcommand{\Y} {\mbox{$Y$}\xspace}
\newcommand{\Z} {\mbox{$Z$}\xspace}

\newcommand{\e} {\mbox{$\mathbf{e}$}\xspace}
\newcommand{\ex} {\mbox{$\mathbf{e_x}$}\xspace}
\newcommand{\ey} {\mbox{$\mathbf{e_y}$}\xspace}
\newcommand{\ez} {\mbox{$\mathbf{e_z}$}\xspace}
\newcommand{\ea} {\mbox{$\mathbf{e_A}$}\xspace}
\newcommand{\etheta} {\mbox{$\mathbf{e_\theta}$}\xspace}
\newcommand{\er} {\mbox{$\mathbf{e}_r$}\xspace}

\newcommand{\ihat} {\mbox{$\mathbf{\hat{i}}$}\xspace}
\newcommand{\jhat} {\mbox{$\mathbf{\hat{j}}$}\xspace}
\newcommand{\khat} {\mbox{$\mathbf{\hat{k}}$}\xspace}
\newcommand{\xhat} {\mbox{$\mathbf{\hat{x}}$}\xspace}
\newcommand{\yhat} {\mbox{$\mathbf{\hat{y}}$}\xspace}
\newcommand{\zhat} {\mbox{$\mathbf{\hat{z}}$}\xspace}

\renewcommand{\i} {\mbox{$\mathbf{\i}$}\xspace}
\renewcommand{\j} {\mbox{$\mathbf{\j}$}\xspace}
\renewcommand{\k} {\mbox{$\mathbf{k}$}\xspace}

\renewcommand{\deg}{\mbox{$^\circ$}\xspace}
\newcommand{\degc}{\mbox{$^\circ$}C\xspace}
\newcommand{\cdeg}{\mbox{ C$^\circ$}\xspace}
\newcommand{\degf}{\mbox{$^\circ$}F\xspace}
\newcommand{\fdeg}{\mbox{ F$^\circ$}\xspace}

\newcommand{\smone}{\mbox{s$^{-1}$}\xspace}
\newcommand{\smtwo}{\mbox{s$^{-2}$}\xspace}
\newcommand{\smthree}{\mbox{s$^{-3}$}\xspace}
\newcommand{\smfour}{\mbox{s$^{-4}$}\xspace}

\newcommand{\BV} {\BlueViolet}
\newcommand{\OG} {\OliveGreen}

\newcommand{\points}[1]{ (#1~points)}
\newcommand{\point}[1]{ (#1~point)}

\newcommand{\kd} {\mbox{$\delta_{ik}$}\xspace}
\newcommand{\emf}{\mbox{${\mathcal E}$}\xspace}

\newcommand{\lw}{\linewidth}

\newenvironment{blist}[1]%
  {\begin{list}{}{\renewcommand{\makelabel}[1]{\textbf{##1}\hfil}%
    \settowidth{\labelwidth}{\textbf{#1}}%
    \setlength{\leftmargin}{\labelwidth+\labelsep}}}%
  {\end{list}}

\newenvironment{nlist}[1]%
 {\begin{list}{}{\renewcommand{\makelabel}[1]{\textrm{##1}\hfil}%
    \settowidth{\labelwidth}{\textrm{#1}}%
    \setlength{\leftmargin}{\labelwidth+\labelsep}}}%
  {\end{list}}
  
\newenvironment{dlist}%
  {\begin{enumerate}[label=\ding{\value*},start=192,itemsep=3ex,labelsep=*,leftmargin=4ex]}%
  {\end{enumerate}}
   
\newenvironment{subpart}[1][1]%
  {\begin{enumerate}[label=(\alph*),align=parleft,itemsep=0.5ex,widest=a,start=#1]}%
  {\end{enumerate}}

\newenvironment{choices}%
  {\begin{enumerate}[label=\Alph*.,align=parleft,itemsep=0.1ex,widest=a]\small}%
  {\end{enumerate}}
  
\newenvironment{MP}{\begin{minipage}[t]{.45\lw}}{\end{minipage}}

\newenvironment{Ventry}[1]%
  {\begin{list}{}{\renewcommand{\makelabel}[1]{\textbf{##1}\hfil}%
    \settowidth{\labelwidth}{\textbf{#1}}%
    \setlength{\leftmargin}{\labelwidth+\labelsep}}}%
  {\end{list}}

\setenumerate{leftmargin=*,labelsep=*,widest=18,itemsep=2ex}


\usepackage{amssymb}
\usepackage{relsize}

%%% BEGIN DOCUMENT

\begin{document}
\thispagestyle{empty}
\raggedright
\normalsize

\vspace*{-10ex}

\begin{minipage}[t]{.4\lw}
PHYS \NUMBER | \TERM\\
\DOCUMENT
\end{minipage}
\hfill
\begin{minipage}[t]{.5\lw}
Name: \uline{\hfill}\\[.2in]
Student ID: \uline{\hfill}
\end{minipage}

\vspace{.1in}

Your TA {\bf (circle one)}:

\begin{tabular}{l|ccc}
Time: & Teddy (section 110) & Trevor (section 111) & Jose (section 112)
\end{tabular}

\vfill

\begin{center}
\framebox{\bf\Large TEST VERSION \VERSION}

\vspace{.1in}

\uline{{\bf\Large Please do not open the exam until you are asked to.}}
\end{center}

\framebox{\begin{minipage}{\textwidth}
\begin{itemize}
\item Write in and bubble in your \uline{{\bf name and student ID number}} on the bubble sheet now.
\item Bubble in the \uline{{\bf TEST VERSION (\VERSION)}} at the center of the bubble sheet now.
\item As you take the exam, \uline{{\bf show all your work}} on the exam and \uline{{\bf circle the correct answers}} on your exam. Your circled answers and bubbled answers must agree.
\item When you finish the exam, place the bubble sheet in the Version \VERSION\ pile and place your booklet in the pile for your recitation section.
\end{itemize}
\end{minipage}
}

\vspace{.15in}

The exam consists of 22 multiple choice questions, worth 1 point each for a total of 22 points. Allowed material: Hand-written equation sheet (2 pieces, front and back, of letter-sized paper), pencils and erasers. Calculators are allowed, but cell phones are not and must be turned off.

\vspace{.15in}

{\bf By handing in this exam, you agree to the following statement: "On my honor, as a University of Colorado Student, I have neither given nor received unauthorized assistance on this work"}

\begin{center}
Signature\uline{\hspace{4in}}
\end{center}

Possibly useful information:\\
\begin{minipage}[t]{.25\lw}
\vspace{0in}
\flushleft
%\includegraphics[height=1.5in]{C:/Users/Daniel/Documents/Teaching/Tools/trig.jpg}
\includegraphics[height=1.5in]{C:/Users/Danie/Documents/trig.jpg}
\end{minipage}\hfill
\begin{minipage}[t]{.7\lw}
\begingroup
\addtolength{\jot}{-1em}
\begin{align*}
g&=9.8\ {\rm m}/{\rm s}^2\\
G&=6.67\times10^{-11}\ \dfrac{\text{N}\cdot\text{m}^2}{\text{kg}^2}
\end{align*}
\endgroup
\end{minipage}

\center{\bf The last page of this booklet is for scratch work.}

\newpage

%%% Begin the Problems

\normalsize

\flushleft{\bf Note: In all questions, unless otherwise stated, neglect air resistance.}\\[.2in]

\begin{enumerate}

%%%%%%%%%%%%%%%%%%%%%%%%%%%%%%%%%%%%%%%%%%%%%%%%%%%%%%%%%%%%%%%%%

\item A book of mass $m$ is raised straight up from the floor to a shelf a distance $h$ above the floor. What was the change in the gravitational potential energy of the Earth-book system?\\
\begin{minipage}[t]{.7\lw}
\begin{choices}
\item $\Delta\text{PE}=+mgh$
\item $\Delta\text{PE}=-mgh$
\item $\Delta\text{PE}=+mh$
\item $\Delta\text{PE}=-mh$
\item $\Delta\text{PE}=0$
\end{choices}
\end{minipage}\hfill
\begin{minipage}[t]{.2\lw}
\vspace{.2in}
\flushright
\includegraphics[width=\lw]{book_lift}
\end{minipage}

\item A fish swims across a pond at $2.0$ m/s. The tail of the fish exerts a force of $0.30$ N in the direction of motion to overcome drag forces exerted on the fish by the water. What is the fish's power output?
\begin{choices}
\item $0.15$ W
\item $0.60$ W
\item $1.2$ W
\item $2.3$ W
\item $6.7$ W
\end{choices}

\item An object is acted on by two forces, each of magnitude $F=5.0$ N as it moves a distance $d=2.0$ m to the right, as shown. Calculate the net work done on the object by the two forces.\\
\begin{minipage}[t]{.45\lw}
\begin{choices}
\item $20$ J
\item $13$ J
\item $10$ J
\item $7.1$ J
\item $5.0$ J
\end{choices}
\end{minipage}\hfill
\begin{minipage}[t]{.5\lw}
\vspace{0in}
\flushright
\includegraphics[width=\lw]{work_calc}
\end{minipage}

\vfill

\item What are the units of the spring constant $k$ that appears in Hooke's Law, $F=-kx$?
\begin{choices}
\item $[k]={\rm N}$
\item $[k]={\rm N}\cdot{\rm m}$
\item $[k]=\dfrac{\rm N}{\rm m}$
\item $[k]=\dfrac{1}{\rm m}$
\item $[k]=\dfrac{1}{{\rm N}\cdot{\rm m}}$
\end{choices}

\newpage

\item A \textit{geostationary} satellite is in the earth's equatorial plane. The speed of the satellite is $v$ and the radius of the orbit is $R$. Your boss tells you that she wants the satellite to move to a new circular orbit with a radius that is \textit{less than} $R$.\\[.1in]
Fill in the blank: The new orbit's speed will be \uline{\hspace{1in}} $v$.\\
\begin{minipage}[t]{.45\lw}
\begin{choices}
\item greater than
\item less than
\item equal to
\end{choices}
\end{minipage}\hfill
\begin{minipage}[t]{.5\lw}
\vspace{0in}
\flushright
\includegraphics[width=\lw]{MP_satellite2}
\end{minipage}

\vfill

\begin{minipage}[t]{.6\lw}
\item A spring is used to launch a block up a frictionless ramp as shown in the figure. At time $t_1$, the spring is compressed by a distance $x$ from its equilibrium and the block is released from point A. The block moves up the ramp, losing contact with the spring at point B. When it passes point C at $t_2$, the block has a speed $v_{\rm C}$ directed up the ramp. The vertical distance between points A and C is $h$. Which equation correctly expresses conservation of energy for the interval from $t_1$ to $t_2$?
\begin{choices}
\item $\dfrac{1}{2}kx^2=\dfrac{1}{2}mv_{\rm C}^2$
\item $\dfrac{1}{2}kx^2=-\dfrac{1}{2}mv_{\rm C}^2+mgh$
\item $\dfrac{1}{2}kx^2=mgh$
\item $\dfrac{1}{2}kx^2=\dfrac{1}{2}mv_{\rm C}^2-mgh$
\item $\dfrac{1}{2}kx^2=\dfrac{1}{2}mv_{\rm C}^2+mgh$
\end{choices}
\end{minipage}\hfill
\begin{minipage}[t]{.35\lw}
\vspace{0in}
\flushright
\includegraphics[width=\lw]{spring_ramp}
\end{minipage}

\vspace{.25in}

\item When a force of 10 N is applied to a certain idea spring it compresses a distance 0.01 m. When the applied force is doubled to 20 N, what happens to the spring constant $k$ of the spring?
\begin{choices}
\item The spring constant increases by a factor of 2.
\item The spring constant increases by a factor of 4.
\item The spring constant decreases by a factor of 2.
\item The spring constant decreases by a factor of 4.
\item The spring constant remains the same.
\end{choices}

\newpage

\begin{minipage}[t]{.6\lw}
\item The gravitational potential energy $U(x)$ and the total energy $E_{\rm tot}$ for a mass sliding on a frictionless surface are shown. The horizontal axis shows position with each square representing $1$ meter. The vertical axis shows energy with each square representing $1$ joule. When the mass is at $x=3$ m, what is its kinetic energy?
\begin{choices}
\item $1$ J
\item $3$ J
\item $7$ J
\item $10$ J
\item $13$ J
\end{choices}
\end{minipage}\hfill
\begin{minipage}[t]{.35\lw}
\vspace{0in}
\flushright
\includegraphics[width=\lw]{PE_graph}
\end{minipage}

\item A small rock of mass $m$ is in circular orbit around a large planet of mass $M$. The force of gravity on the rock from the planet has magnitude $F_\text{on rock}$. The force of gravity on the planet from the rock has magnitude $F_\text{on planet}$. What is the ratio of the magnitudes of the forces?\\
\begin{minipage}[t]{.65\lw}
\begin{choices}
\item $\dfrac{F_\text{on rock}}{F_\text{on planet}}=1$
\item $\dfrac{F_\text{on rock}}{F_\text{on planet}}=m/M$
\item $\dfrac{F_\text{on rock}}{F_\text{on planet}}=M/m$
\item $\dfrac{F_\text{on rock}}{F_\text{on planet}}=(m/M)^2$
\item $\dfrac{F_\text{on rock}}{F_\text{on planet}}=(M/m)^2$
\end{choices}
\end{minipage}\hfill
\begin{minipage}[t]{.3\lw}
\vspace{0in}
\flushright
\includegraphics[width=\lw]{rock_orbit}
\end{minipage}

\item An object of mass $m$, attached to a spring with spring constant $k$, is pulled horizontally across a horizontal, frictionless surface by a force of magnitude $F$ and the resulting acceleration has magnitude $a$. What is $\Delta x$, the amount by which the spring is stretched beyond its equilibrium length?\\
\begin{minipage}[t]{.45\lw}
\begin{choices}
\item $\Delta x=ma$
\item $\Delta x=mak$
\item $\Delta x=\dfrac{F}{m}$
\item $\Delta x=\dfrac{k}{ma}$
\item $\Delta x=\dfrac{ma}{k}$
\end{choices}
\end{minipage}\hfill
\begin{minipage}[t]{.5\lw}
\vspace{0in}
\flushright
\includegraphics[width=\lw]{spring_pull}
\end{minipage}

\begin{minipage}[t]{.6\lw}
\item Two identical blocks, A and B, on a surface of negligible friction are connected by a spring of negligible mass. The spring is initially unstretched. During the interval from $t_1$ to $t_2$, block A is pushed through a distance $d_A$ by a hand exerting a force of magnitude $F_A$, as shown. Block B is held in place by a wall. The wall exerts a force on block B that varies with time but is always directed to the left. During the interval from $t_1$ to $t_2$, the net work done on block B by the wall \ldots
\begin{choices}
\item is zero.
\item is positive.
\item is negative.
\item depends on which direction is chosen as positive.
\end{choices}
\end{minipage}\hfill
\begin{minipage}[t]{.35\lw}
\vspace{0in}
\flushright
\includegraphics[width=\lw]{work_wall}
\end{minipage}

\vfill

\item A car of mass $m$ travels around a banked turn of radius $R$ at an angle $\theta$ to the horizontal. The surface is rough and there is friction. The car travels at a steady speed $v$ and is not on the verge of slipping.
\begin{center}
\includegraphics[height=2in]{banked_curve}
\end{center}
\uline{Choose the non-tilted coordinate system shown in the figure.} Which set of $x$ and $y$ components are correct for the acceleration vector of the car?
\begin{choices}
\item $a_x=+\dfrac{v^2}{R}\cos\theta$ and $a_y=+\dfrac{v^2}{R}\sin\theta$\\[.1in]
\item $a_x=+\dfrac{v^2}{R}\sin\theta$ and $a_y=+\dfrac{v^2}{R}\cos\theta$\\[.1in]
\item $a_x=+\dfrac{v^2}{R}$ and $a_y=-g$\\[.1in]
\item $a_x=+\dfrac{v^2}{R}$ and $a_y=0$\\[.1in]
\item $a_x=0$ and $a_y=-g$
\end{choices}

\newpage

\item Two carts, A and B, are initially at rest on a horizontal frictionless table as shown in the top-view diagram below. A constant force of magnitude $F_0$ is exerted on each cart as it travels between two marks on the table. Cart B has greater mass than cart A.
\begin{center}
\includegraphics[height=1.5in]{tut_work_carts}
\end{center}
Let ${\rm KE}_{\rm A}$ be the kinetic energy of cart A when it crosses the second mark and let ${\rm KE}_{\rm B}$ be the kinetic energy of cart B when it crosses the second mark. Which statement is correct?
\begin{choices}
\item ${\rm KE}_{\rm A}>{\rm KE}_{\rm B}$
\item ${\rm KE}_{\rm A}<{\rm KE}_{\rm B}$
\item ${\rm KE}_{\rm A}={\rm KE}_{\rm B}$
\end{choices}

\item A ball of mass $m$ is thrown straight up. Consider the time interval from just after release until the moment when the ball reaches a maximum height $h$. What is the work done on the ball by the force of gravity?
\begin{choices}
\item $0$
\item $-mgh$
\item $+mgh$
\item Depends on whether up or down is chosen as the positive direction
\end{choices}

\begin{minipage}[t]{.5\lw}
\item Compute the dot product $\vec{\bf v}\cdot\vec{\bf w}$ to the nearest $0.1$.
\begin{choices}
\item $\vec{\bf v}\cdot\vec{\bf w}=0$
\item $\vec{\bf v}\cdot\vec{\bf w}=+1$
\item $\vec{\bf v}\cdot\vec{\bf w}=-1.4$
\item $\vec{\bf v}\cdot\vec{\bf w}=+0.7$
\item $\vec{\bf v}\cdot\vec{\bf w}=-0.7$
\end{choices}
\end{minipage}\hfill
\begin{minipage}[t]{.45\lw}
\vspace{0in}
\flushright
\includegraphics[width=\lw]{dot_product}
\end{minipage}

\newpage

\item A particle of mass $m$ slides along a frictionless track as shown in the figure, starting with a speed $v_0$ from position A. The three hills have heights $8h$, $5h$, and $3h$ as shown. Find an expression for the kinetic energy of the particle when it is at position B.\\
\begin{minipage}[t]{.4\lw}
\begin{choices}
\item ${\rm KE}_{\rm B}=5mgh$
\item ${\rm KE}_{\rm B}=\dfrac{1}{2}mv_0^2+5mgh$
\item ${\rm KE}_{\rm B}=3mgh$
\item ${\rm KE}_{\rm B}=\dfrac{1}{2}mv_0^2+3mgh$
\item ${\rm KE}_{\rm B}=\dfrac{1}{2}mv_0^2-3mgh$
\end{choices}
\end{minipage}\hfill
\begin{minipage}[t]{.55\lw}
\vspace{0in}
\flushright
\includegraphics[width=\lw]{coaster}
\end{minipage}

\vfill

\item Two particles are separated by a center-to-center distance $r$. The force of gravitational attraction between them has magnitude $F$. Now the center-to-center distance between the particles is increased to $10r$. At this new separation, what is the magnitude of the force of gravitational attraction between the particles? Fill in the blank: $F_\text{new}=\uline{\hspace{.5in}}\cdot F$
\begin{choices}
\item $\dfrac{1}{100}$\\[.1in]
\item $\dfrac{1}{10}$\\[.1in]
\item $\dfrac{1}{\sqrt{10}}$\\[.1in]
\item $10$\\[.1in]
\item $100$
\end{choices}

\item Two skiers arrive at the bottom of a ramp with the same speed. Friction may be ignored in this problem. One skier is an adult and one is a child. The adult has 3 times the mass of the child. The skiers slide up the ramp and eventually stop, reaching some maximum height. The adult's maximum height will is $\uline{\hspace{1in}}$ the child's maximum height.
\begin{choices}
\item one ninth
\item one third
\item the same as
\item three times
\item nine times
\end{choices}

\vspace{.5in}

\newpage

\item Three masses ($2m$, $3m$, and $m$) are placed in a row in deep space (no nearby planets or stars). Each mass is a center-to-center distance $d$ from its neighbor as shown. What is the magnitude of the net force on the middle mass $3m$?\\
\begin{minipage}[t]{.5\lw}
\begin{choices}
\item $\dfrac{3Gm^2}{2d^2}$
\item $\dfrac{7Gm^2}{4d^2}$
\item $\dfrac{3Gm^2}{d^2}$
\item $\dfrac{6Gm^2}{d^2}$
\item $\dfrac{7Gm^2}{d^2}$
\end{choices}
\end{minipage}\hfill
\begin{minipage}[t]{.45\lw}
\vspace{0in}
\flushright
\includegraphics[width=\lw]{3masses}
\end{minipage}

\vspace{.75in}

\item A dancer of mass $m$ is standing on one leg on a drawbridge that is tilted upward. The coefficients of static and kinetic friction between the drawbridge and the dancer's foot are $\mu_s$ and $\mu_k$, respectively. $\vec{n}$ represents the normal force exerted on the dancer, $\vec{F}_g$ represents the gravitational force exerted on the dancer, and $\vec{f}$ represents the frictional force exerted on the dancer. The bridge is at a slight angle $\theta$ as shown. The dancer is standing perfectly still and the angle is small enough that the dancer is not on the verge of slipping. Find an expression for the magnitude of the friction force.\\
\begin{minipage}[t]{.45\lw}
\begin{choices}
\item $f=mg\sin\theta$
\item $f=mg\cos\theta$
\item $f=\mu_smg\cos\theta$
\item $f=\mu_smg$
\item $f=mg$
\end{choices}
\end{minipage}\hfill
\begin{minipage}[t]{.5\lw}
\vspace{0in}
\flushright
\includegraphics[width=\lw]{dancer}
\end{minipage}

\newpage

\item Two planets orbit a distant star, both in perfectly circular orbits of the same radius. Planet A has one third the mass of planet B. Compare the magnitudes of the accelerations of these planets.
\begin{choices}
\item $\dfrac{a_A}{a_B}=\dfrac{1}{9}$
\item $\dfrac{a_A}{a_B}=\dfrac{1}{3}$
\item $\dfrac{a_A}{a_B}=1$
\item $\dfrac{a_A}{a_B}=3$
\item $\dfrac{a_A}{a_B}=9$
\end{choices}

\item A block of mass $m$ is given an initial speed and then it slides across a rough horizontal surface (coefficient of kinetic friction $\mu_k$). Nothing is in contact with the block except for the surface. If the box travels a distance $d$ before coming to rest, what was the initial speed?\\
\begin{minipage}[t]{.5\lw}
\begin{choices}
\item $v_0=\dfrac{d}{\mu_kmg}$\\[.025in]
\item $v_0=\dfrac{\mu_kmg}{d}$\\[.025in]
\item $v_0=d+\mu_kmg$\\[.025in]
\item $v_0=\sqrt{2\mu_kmg}$\\[.025in]
\item $v_0=\sqrt{2\mu_kgd}$
\end{choices}
\end{minipage}\hfill
\begin{minipage}[t]{.45\lw}
\vspace{0in}
\flushright
\includegraphics[width=\lw]{carpet}
\end{minipage}

%%%%%%%%%%%%%%%%%%%%%%%%%%%%%%%%%%%%%%%%%%%%%%%%%%%%%%%%%%%%%%%%%

\end{enumerate}

\newpage
\thispagestyle{empty}
{\footnotesize (Use this page if you need extra space for your calculations)}
%\newpage
%\thispagestyle{empty}
%\vbox{}

\end{document}